% Quantized light
% g2 (second order correlation function)
% Fock state, Coherent state, Thermal state
% single photon, laser, thermal light
% how to proob: HBT and GRA experiment
\chapter{Theory}
\minitocmargin
This is theory

\section{Theory on quantized light}
\subsection{Fock, coherent, thermal state of light}
양자에서 생성연산자, 소멸연산자 했던거 기억 나지? 그것처럼 여기서도 똑같은 짓을 광자의 개수를 state로 삼아서 한다. 우리가 앞으로 볼 state는 self 내적이 확률밀도함수인 입자의 파동함수나, 전기장이나 자기장의 크기를 직접적으로 나타내는 전자기학의 waveguide mode 같이 공간적인 분포를 나타내는 것이 아니고 이 때 광자가 단위 시간 당 몇 개가 나오는가?를 보는 것이라 보면 된다.
\paragraph{Fock state}
Fock state에 대한 대략적인 설명. 대충 넘버 오퍼레이터의 고유상태가 어쩌구..아주 중요한 것은 fock state of |1>일 때를 싱글포톤이라고 한다는 것!
\paragraph{Coherent state}
마찬가지로 coherent state에 대한 설명. 이건 anihilation op.의 고유상태이며 Fock state의 lin.comb이고 레이저가 여기 해당된다고 강조. 푸아송 분포, 랜덤, SPDC가 여기 해당
\paragraph{Thermal state}
마찬가지
\subsection{g$^{(2)}$: state의 검증 방법}
g2는 뭐냐면.. 어쩌구.. 그리고 저 state들이 각각 anti bunching, random, bunching of photon을 나타낸다는 것을 명시. 그리고 간단한 계산도.
\paragraph{번칭 관련 설명}
\paragraph{실제 계산 for 2 detector}
\paragraph{실제 계산 for 3 detector}
그리고 accidental coincidence 이야기도 할 것
\subsection{quantum description of beam splitter}
ㅇㅇ


\section{HBT 실험}
대충 설명
\subsection{싱글포톤과 heralded 싱글포톤의 차이(짱중요)}

\section{GRA 실험}
ㅇㅇ

\section{마이컬슨 간섭계}
ㅇㅇ. 여기 더해서 coherence와 envelope, 광원별로 뭐가 어찌 되는지(LED는 coherence가 짧아서 envelope이 어쩌구..)

\section{양자지우개}