\documentclass[symmetric]{tufte-book}

\title{Guide Manual for\\Senior Physic\\Laboratory:\\ Quantum Optics}
\author{SNU Dept. of Phys.}
\publisher{Kim Heeju, Kim Jungjun, Park Youngmin}


%%
% Just some sample text
\usepackage{lipsum}
\usepackage{kotex}


%%
% For graphics / images
\usepackage{graphicx}
\setkeys{Gin}{width=\linewidth,totalheight=\textheight,keepaspectratio}
\graphicspath{{graphics/}}

% 헤더 만들기
    \usepackage{fancyhdr}
    % (1) 막대/폰트/간격을 한 번에 바꾸기 위한 사용자 매크로
    \newcommand{\headfontsize}{\normalsize}      % 헤더 글자 크기
    \newcommand{\headsepbarheight}{3ex}        % 막대 높이
    \newcommand{\headsepbarwidth}{1pt}         % 막대 두께
    \newcommand{\headsepbarcolor}{black!70}      % 막대 색
    \newcommand{\headbar}{%
      {\color{\headsepbarcolor}\rule{\headsepbarwidth}{\headsepbarheight}}}
    % (2) \leftmark(챕터), \rightmark(섹션)에 "번호 + 제목"을 넣도록 재정의
    \makeatletter
    \renewcommand{\chaptermark}[1]{%
      \markboth{\thechapter\enspace #1}{}}
    \renewcommand{\sectionmark}[1]{%
      \markright{\thesection\enspace #1}}
    \makeatother
    % (3) fancyhdr로 양면 헤더 구성
    \pagestyle{fancy}
    \fancyhf{}                               % 기존 헤더/푸터 비우기
    \renewcommand{\headrulewidth}{0pt}       % 기본 가로줄 제거
    % 짝수(왼쪽) 페이지 = [바깥쪽 헤더] 챕터  |  페이지
    \fancyhead[LE]{%
      \headfontsize
      \thepage\enspace\headbar\enspace\nouppercase{\leftmark}}
    % 홀수(오른쪽) 페이지 = [바깥쪽 헤더] 섹션  |  페이지
    \fancyhead[RO]{%
      \headfontsize
      \nouppercase{\rightmark}\enspace\headbar\enspace\thepage}


% 맨 앞에 표지 세팅
    \makeatletter
    \renewcommand{\maketitlepage}{%
    \begingroup%
    \setlength{\parindent}{0pt}
    {\fontsize{24}{24}\selectfont\textit{\@author}\par}
    \vspace{1.75in}{\fontsize{36}{54}\selectfont\@title\par}
    \vspace{0.5in}{\fontsize{14}{14}\selectfont\textsf{\smallcaps{\@date}}\par}
    \vfill{\fontsize{14}{14}\selectfont\textit{\@publisher}\par}
    \thispagestyle{empty}
    \endgroup
    }
    \makeatother

% 목차 설정
    \setcounter{tocdepth}{2}% 목차에 섹션까지 표시
    \setcounter{secnumdepth}{2} % 섹션까지 넘버링 넣기
    
    %% 목차에 쓸 폰트 매크로
    \newcommand{\tocchapfont}{\bfseries\large\rmfamily}   % 챕터 = 굵게 + large
    \newcommand{\tocsecfont}{\normalsize\rmfamily}        % 섹션 = 보통 크기
    \newcommand{\tocpagenumfont}{\normalfont\rmfamily}    % 페이지 번호 폰트
    
    %% 챕터 (TOC에서 들여쓰기 없음)
    \titlecontents{chapter}%
      [0em]                                   % 챕터는 들여쓰기 0
      {\tocchapfont}                          % 전체 항목 서식
      {\contentslabel{2.3em}}                 % 번호 자리 폭
      {}                                      % 라벨 없을 때
      {\hfill\tocpagenumfont\contentspage}    % 페이지 번호
      [\addvspace{0.25\baselineskip}]         % 항목 뒤 간격
    %% 섹션 (TOC에서 들여쓰기)
    \titlecontents{section}%
      [2.5em]                                 % ← 섹션 들여쓰기 (원하는 만큼)
      {\tocsecfont}                           % 섹션 항목 폰트/크기
      {\contentslabel{3.0em}}                 % "1.2" 같은 번호 폭
      {\hspace*{3.0em}}                       % 라벨 없을 때 여백
      {\hfill\tocpagenumfont\contentspage}    % 페이지 번호
      [\vspace{0.15\baselineskip}]            % 항목 뒤 간격

% 챕터마다 목차 넣는거
\usepackage{minitoc}
    \setcounter{minitocdepth}{3}      % 섹션까지 표시 (필요시 3=서브섹션)
    \dominitoc
    \renewcommand{\mtctitle}{}   % 제목 제거
    \makeatletter
    \renewcommand{\mtc@rule}{}   % 가로줄 제거
    \makeatother
    \renewcommand{\mtcfont}{\footnotesize\rmfamily}
    \renewcommand{\mtcSfont}{\footnotesize\rmfamily}
    \newcommand{\minitocmargin}{
          \marginnote{
            \begin{minipage}{\marginparwidth}
              \footnotesize\rmfamily
              \minitoc
            \end{minipage}
          }
        }

% Typesets the font size, leading, and measure in the form of 10/12x26 pc.
\newcommand{\measure}[3]{#1/#2$\times$\unit[#3]{pc}}



\begin{document}

\maketitle

% contents
\tableofcontents

\chapter{Introduction}
\minitocmargin
This is Introduction


\section{실험 목적}
본 실험의 목적은 양자광학적 지식을 바탕으로 nonlinear process인 SPDC에서 나온 싱글 포톤을 검증(추후 다루겠지만 엄밀히 말해서 싱글포톤은 아니다. heralded single photon..)을 하는 것과 그걸로 SPI를 해보는 것, 그에 이어 추가적으로 양자 지우개 실험을 하는 것이다. 추후 예산이 허락한다면 대표적인 양자컴퓨터의 quantum supremency를 보여주는 초기 양자 알고리즘인 도이치-조자 알고리즘도 보여주고, coherence length도 보고, HOM도 하고 entangled photon을 이용해 벨 부등식의 다른 버전인 CHSH 부등식이 깨진다는 것을 보여 양자역학은 실재성과 국소성을 동시에 가지지 않음을 확인할 수도 있다. \cite{Tufte2006}


\section{부품 설명}
\subsection{Light source}
\paragraph{405 nm pump laser \& mount \& driver}
405 nm 레이저 다이오드와 그게 들어가서 전력 공급 및 온도 안정화 등을 책임져주는 레이저 마운트, 그리고 레이저가 collimate 되어서 나올 수 있도록 해주는 렌즈가 한 덩어리를 이루고 거기에 전력 공급을 제어해주는 드라이버로 구성돼있다.
\paragraph{635 nm align laser}
405 nm 레이저로는 align하는 것이 거의 불가능하기 때문에 align을 할 때 필요한 레이저
\paragraph{660 nm align LED}
마이컬슨 간섭계를 얼라인할 때 필요한 도구. 635 nm align laser는 coherence length가 길어 대강 맞춰도 잘 맞아보이는 것 같아보이지만 사실 그렇지 않기 때문에 coherence length가 짧아 더욱 민감하고 정밀하게 align할 수 있게 도와주는 도구이다.

\subsection{BBO crystal, Axicon, colored glass filter}
\paragraph{BBO crystal}
SPDC를 일으키는 핵심 물질. 405 nm 펌프 레이저를 맞아 810 nm의 SPDC cone을 발생시키며, 이 때 cone의 중심축과 빛의 각도는 약 3 $^\circ$이다.
\paragraph{Axicon}
405 nm 레이저는 직접 보면 알겠지만 잘 눈에 보이지 않아서 얼라인에 적합하지 않다. 따라서 align laser와 405 nm laser의 광경로를 iris 등을 이용하여 완전히 일치시킨 다음 align laser를 이용하여 align하는 것이다. 이 때 문제가 있는데, BBO crystal에 의해 발생하는 빛의 경우 원뿔 모양으로 나간다. 따라서 그 경로를 똑같이 복사하기 위해 정확히 같은 각도로 원뿔 모양 cone이 생기게 하는 axicon을 사용하여 SPDC cone을 복제한다고 생각하면 된다.
\paragraph{Colored glass filter}
후술하겠지만, align laser만으로는 완벽한 얼라인이 되지 않는다. 디텍터에 렌즈 튜브를 껴서 SPDC 신호를 전부 모아 약 1 mm$^2$ 가량의 작은 실리콘 칩에 모아야 하는데, 이 때 렌즈 튜브의 초점은 적절한지, align은 잘 되었는지를 align laser만으로는 알기 어렵기 때문에 405 pump laser를 맞았을 때 colored glass filter에서 발생하는 형광을 통해 알아보는 것이다.

\subsection{Detector and time tagger}
\paragraph{SPDMA}
SPDMA 디텍터에 렌즈 튜브 뭉탱이를 껴둔 것. 한 디텍터에 초당 8 M 이상이 들어가지 말게 하고 모든 것 합이 10 M이 넘지 않도록 할 것
\paragraph{Time tagger}
Swabian 사의 Time tagger 20 Thorlabs edu kit edition. 그냥 타임태거 20이라고 헷갈릴 수 있지만, 이건 교육용으로 제작된 것이라 스펙이 훨씬 나쁘다. 여기서 사용하는 타임태거는 jitter가 <720 ps임.

\subsection{빔 스플리터와 간섭계}
\paragraph{빔 스플리터}
이거 그지같은게 빔이 상하방향으로 기울어져있음. 주의.
\paragraph{간섭계}
이건 스테이지에 올려서 쓴다. 이것저것 이렇게 올리면 됨. 스테이지랑 기타 등등이랑..

\section{Set up 테크닉}
\subsection{Mirror mount 만지는 방법}
이 서브섹션에서 다룰건 1. 위 knob 만지면 세로 조정, 아래 knob는 가로 조정 2. 거울 두 개 쓸 때 위 knob 두 개 동시에 반대로 돌라면 beam translation, 아래 knob는 같은 방향으로 돌릴 때 translation이 일어난다는 것 3. 가급적이면 거울의 가운데를 맞추기 4. knob를 위치 0이 아니라 너무 많이 돌려두면 비틀어지는게 좀 커져서 knob 하나만 돌려도 가로 세로 섞여서 돌아가니 가급적이면 너무 많이 돌리지 않는걸 추천한다 를! 다룰 것.
\subsection{back reflection}
백리플렉션은정말중요합니다이게없다면우리는실험을할수없어. 심지어 간섭계랑 렌즈 백리플랙션은 점이 2개씩 생기고 그래서 더 빡셈.
\subsection{기타}
반드시 닭발의 중간쯤에 볼트를 박고, 닭발에는 반드시 와셔를 끼고 써라. 그렇지 않다면 어제 맞춰둔 얼라인이 내일 오니 틀어져있는 것을 확인할 수 있을 것이다
% Quantized light
% g2 (second order correlation function)
% Fock state, Coherent state, Thermal state
% single photon, laser, thermal light
% how to proob: HBT and GRA experiment
\chapter{Theory}
\minitocmargin
This is theory

\section{Theory on quantized light}
\subsection{Fock, coherent, thermal state of light}
양자에서 생성연산자, 소멸연산자 했던거 기억 나지? 그것처럼 여기서도 똑같은 짓을 광자의 개수를 state로 삼아서 한다. 우리가 앞으로 볼 state는 self 내적이 확률밀도함수인 입자의 파동함수나, 전기장이나 자기장의 크기를 직접적으로 나타내는 전자기학의 waveguide mode 같이 공간적인 분포를 나타내는 것이 아니고 이 때 광자가 단위 시간 당 몇 개가 나오는가?를 보는 것이라 보면 된다.
\paragraph{Fock state}
Fock state에 대한 대략적인 설명. 대충 넘버 오퍼레이터의 고유상태가 어쩌구..아주 중요한 것은 fock state of |1>일 때를 싱글포톤이라고 한다는 것!
\paragraph{Coherent state}
마찬가지로 coherent state에 대한 설명. 이건 anihilation op.의 고유상태이며 Fock state의 lin.comb이고 레이저가 여기 해당된다고 강조. 푸아송 분포, 랜덤, SPDC가 여기 해당
\paragraph{Thermal state}
마찬가지
\subsection{g$^{(2)}$: state의 검증 방법}
g2는 뭐냐면.. 어쩌구.. 그리고 저 state들이 각각 anti bunching, random, bunching of photon을 나타낸다는 것을 명시. 그리고 간단한 계산도.
\paragraph{번칭 관련 설명}
\paragraph{실제 계산 for 2 detector}
\paragraph{실제 계산 for 3 detector}
그리고 accidental coincidence 이야기도 할 것
\subsection{quantum description of beam splitter}
ㅇㅇ


\section{HBT 실험}
대충 설명
\subsection{싱글포톤과 heralded 싱글포톤의 차이(짱중요)}

\section{GRA 실험}
ㅇㅇ

\section{마이컬슨 간섭계}
ㅇㅇ. 여기 더해서 coherence와 envelope, 광원별로 뭐가 어찌 되는지(LED는 coherence가 짧아서 envelope이 어쩌구..)

\section{양자지우개}
\chapter{HBT and GRA experiment}
\minitocmargin
\section{배경 이론}
와 실험!
\chapter{Michelson Interferometer and Quantum Eraser}
\minitocmargin
asdafaas 뭔가 테스트용 글


\backmatter

\bibliography{reference}
\bibliographystyle{plainnat}

\end{document}

